%%=============================================================================
%% Inleiding
%%=============================================================================

\chapter{\IfLanguageName{dutch}{Inleiding}{Introduction}}%
\label{ch:inleiding}

In de jaren 80 en 90 werden expertsystemen ingezet als een vorm van kunstmatige intelligentie. Deze systemen simuleerden het besluitvormingsproces van menselijke experts binnen een probleemdomein en bestonden uit een kennisbank met feiten en regels, gecombineerd met een inferentie-engine~\autocite{Angeli2010,Castillo2012}. Dit laatste onderdeel werkte met if-then-regels en kon logische conclusies trekken op basis van de beschikbare kennis~\autocite{Buchanan1988}. Een nadeel van expertsystemen was dat hun denkproces vastlag. Ze konden enkel beslissingen nemen op basis van vooraf vastgelegde regels en logica. Ze waren niet in staat om nieuwe of onverwachte situaties op te vangen of zich aan te passen zoals een mens dat zou doen~\autocite{Castillo2012}. Bovendien had de gebruiker zelf geen vrijheid in het maken van keuzes, maar werd hij slechts begeleid naar een oplossing.\par
\medskip
Halverwege de jaren 90 verschenen de webshops~\autocite{Chu2007}. Het aantal B2C-online winkels blijft tot op de dag van vandaag groeien~\autocite{Pokki2016}. In tegenstelling tot expertsystemen hebben gebruikers de mogelijkheid om zelf te zoeken naar wat ze nodig hebben. Dit geeft hen meer controle en vrijheid in hun keuzeproces, maar biedt tegelijkertijd minder begeleiding dan een expertsysteem.\par
\medskip
Het online winkelen laat producenten en retailers toe meer opties aan te bieden dan voorheen mogelijk was. Dit brengt echter een probleem met zich mee voor gebruikers, die nu het volledige productaanbod moeten analyseren om te bepalen wat ze daadwerkelijk nodig hebben. Om hen hierbij te helpen, worden recom\-mender- of aanbevelingssystemen ingezet~\autocite{Sivapalan2014}. De afgelopen jaren zijn deze informatiefilteringssytemen op grote schaal toegepast in e-com\-merce toepassingen om de voorkeur van een gebruiker voor een item te voorspellen~\autocite{Thorat2015}. Deze suggesties, op basis van de behoeften van een gebruiker, spelen een rol in zijn of haar besluitvormingsproces.\par

Naast het creëren van een aangename en correcte gebruikerservaring, spelen visuele effecten tijdens het navigeren op een website een significante rol in de communicatie van inhoud. Een woord dat getoond wordt, de kleur die daaraan wordt toegekend, de lettergrootte die wordt gekozen…~\autocite{Bordbar2016}. Al deze keuzes communiceren iets naar de bezoeker van de site. Volgens~\textcite{Lee2012} is de bruikbaarheid van een website een fundamenteel onderdeel van de gebruikerservaring.

\section{\IfLanguageName{dutch}{Probleemstelling}{Problem Statement}}%
\label{sec:probleemstelling}
Deze studie richt zich op de online webIDE van het IDP-Z3 redeneersysteem van de KU Leuven. De huidige gebruikersinterface van deze webIDE overrompelt eindgebruikers met ongestructureerde componenten en informatie. De gebruiker wordt niet begeleid of geholpen doorheen de UI om te vinden wat hij of zij zoekt. Momenteel wordt er een set van eigenschappen (properties) gevisualiseerd die kan gezien worden als filters die niet aangepast wordt door interactie van de gebruiker. Er is ruimte voor het verbeteren in bruikbaarheid en navigeerbaarheid van de user interface. Daarnaast heeft de huidige UI een zwakke visuele uitstraling.

\section{\IfLanguageName{dutch}{Onderzoeksvraag}{Research question}}%
\label{sec:onderzoeksvraag}
De centrale vraag van dit onderzoek is: Hoe kan een front-end softwarebibliotheek die een dataset op een interactieve, dynamische en prioriteit gestuurde manier visualiseert, geïmplementeerd worden om structuur en duidelijkheid te creëren, zodat de navigatie en bruikbaarheid van de IDP-Z3 webIDE van de KU Leuven verbeterd worden? Verder worden er antwoorden gevonden op de volgende deelvragen:
\begin{itemize}
    \item Op welke manier wordt de dataset in de huidige gebruikersinterface gevisualiseerd?
    \item Welke problemen heeft de huidige gebruikersinterface?
    \item Welke visuele elementen spelen een rol in het verbeteren van de gebruikerservaring?
    \item Wat zijn de belangrijkste vereisten voor de nieuwe gebruikersinterface?
    \item Wat zijn de visualisatietools die kunnen gebruikt worden en welke is de meest toepasselijke?
\end{itemize}

\section{\IfLanguageName{dutch}{Onderzoeksdoelstelling}{Research objective}}%
\label{sec:onderzoeksdoelstelling}
Het doel van deze studie is het ontwikkelen van een dynamische en interactieve gebruikersinterface (UI) voor de Interactive Consultant (IC) van de KU Leuven. Deze UI visualiseert een dynamische set van properties, die worden geïdentificeerd door korte woorden of zinnen. Hierbij is het essentieel dat het belang van deze eigenschappen wordt berekend en weergegeven op basis van gebruikersbehoeften en de volgorde waarin ze worden geselecteerd.\smallskip\par Om dit te realiseren, wordt er een gepersonaliseerde, tweede generatie recomm\-ender-systeem ingezet. De aanbevelingen zijn gebaseerd op gebruikersvoorkeuren en -gedrag, maar in tegenstelling tot bestaande systemen worden de suggesties niet opgelegd. Het systeem is ontworpen om gebruikers niet het gevoel te geven dat ze worden beïnvloed door reclame of opdringerige aanbevelingen. De gebruiker krijgt vrijblijvende begeleiding, maar behoudt tegelijkertijd volledige vrijheid in het maken van keuzes. Daarnaast moet de nieuwe UI, net als de huidige, in staat zijn om feedback te geven over properties.\smallskip\par  Voorafgaand aan de implementatie wordt een vergelijkende studie uitgevoerd, gevolgd door de ontwikkeling van een proof-of-concept. Het beoogde eindresultaat is een webapplicatie gebruikmakend van een softwarebibliotheek en een berekeningsmodel gebaseerd op een memory-based algoritme. Deze applicatie communiceert met een REST API-server en slaat de selectiekeuzes van gebruikers op in een databank. Op basis van een requirements-analyse uit de vergelijkende studie wordt bepaald welke front-end softwarebibliotheek hiervoor het meest geschikt is.

\section{\IfLanguageName{dutch}{Opzet van deze bachelorproef}{Structure of this bachelor thesis}}%
\label{sec:opzet-bachelorproef}

De rest van deze bachelorproef is als volgt opgebouwd:

In Hoofdstuk~\ref{ch:stand-van-zaken} wordt een overzicht gegeven van de stand van zaken binnen het onderzoeksdomein, op basis van een literatuurstudie.

In Hoofdstuk~\ref{ch:methodologie} wordt de methodologie toegelicht en worden de gebruikte onderzoekstechnieken besproken om een antwoord te kunnen formuleren op de onderzoeksvragen.

In Hoofdstuk~\ref{ch:conclusie}, tenslotte, wordt de conclusie gegeven en een antwoord geformuleerd op de onderzoeksvragen. Daarbij wordt ook een aanzet gegeven voor toekomstig onderzoek binnen dit domein.
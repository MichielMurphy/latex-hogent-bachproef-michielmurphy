%%=============================================================================
%% Samenvatting
%%=============================================================================

\chapter*{\IfLanguageName{dutch}{Samenvatting}{Abstract}}
Websites bestaan sinds de jaren 90 en zijn sindsdien onderdeel van het dagelijks leven geworden. Wereldwijd worden dagelijks meer dan een miljard websites bezocht, waarbij gebruikers op zoek zijn naar informatie, diensten of producten. Echter, niet elke bezoeker slaagt erin de gewenste informatie snel en effectief te vinden. De manier waarop informatie wordt gepresenteerd en hoe gebruikers door een website navigeren, heeft directe invloed op de algemene gebruikerservaring en de effectiviteit van de website. Deze studie focust zich op de gebruikersinterface van de webIDE van het IDP-Z3 redeneersysteem van de KU Leuven, die gekarakteriseerd wordt door een onaantrekkelijke en ongestructureerde interface, wat de bruikbaarheid bemoeilijkt. Daarnaast wordt er geen visueel onderscheid gemaakt tussen relevante en minder relevante informatie, wat de navigeerbaarheid negatief beïnvloedt. Deze studie wil dit probleem aanpakken door de huidige gebruikersinterface te verbeteren. Dit onderzoek wordt uitgevoerd om de volgende vraag te beantwoorden: Hoe kan een front-end softwarebibliotheek die een dataset op een interactieve, dynamische en prioriteit gestuurde manier visualiseert, geïmplementeerd worden om structuur en duidelijkheid te creëren, zodat de navigeerbaarheid en bruikbaarheid van de IDP-Z3 webIDE van de KU Leuven verbeterd worden? Er wordt een vergelijkende studie uitgevoerd om de beschikbare visualisatietools in kaart te brengen en de meest geschikte softwarebibliotheek te identificeren. Vervolgens wordt er een prototype van een webapplicatie ontwikkeld om de haalbaarheid van de implementatie van de visualisatie te testen en de invloed ervan te evalueren. De resultaten van dit onderzoek tonen aan dat het haalbaar is om met een front-end softwarebibliotheek een interactieve, dynamische en prioriteitsgestuurde interface te ontwikkelen. De gekozen bibliotheek en visualisatietechniek slagen erin de bruikbaarheid en navigeerbaarheid van de applicatie te verbeteren. Hieruit kan geconcludeerd worden dat zowel de gebruikerservaring als de efficiëntie positief worden beïnvloed.
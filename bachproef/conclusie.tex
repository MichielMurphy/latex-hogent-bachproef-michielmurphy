%%=============================================================================
%% Conclusie
%%=============================================================================

\chapter{Conclusie}%
\label{ch:conclusie}

% TODO: Trek een duidelijke conclusie, in de vorm van een antwoord op de
% onderzoeksvra(a)g(en). Wat was jouw bijdrage aan het onderzoeksdomein en
% hoe biedt dit meerwaarde aan het vakgebied/doelgroep? 
% Reflecteer kritisch over het resultaat. In Engelse teksten wordt deze sectie
% ``Discussion'' genoemd. Had je deze uitkomst verwacht? Zijn er zaken die nog
% niet duidelijk zijn?
% Heeft het onderzoek geleid tot nieuwe vragen die uitnodigen tot verder 
%onderzoek?

Dit onderzoek toont aan dat het haalbaar is om informatie op een interactieve, dynamische en prioriteitsgestuurde manier te visualiseren met behulp van een front-end softwarebibliotheek. Een woordwolk biedt een efficiënte manier om een compacte en intuïtieve gebruikersinterface te ontwikkelen. De styling van deze visualisatie, zoals kleuren en tekstgroottes, verbeteren de navigeerbaarheid en bruikbaarheid van de interface.\medskip\par De vergelijkende studie geeft een duidelijk overzicht van de beschikbare visualisatietools. Aangezien elk alternatief voor- en nadelen heeft, is het moeilijk om slechts één ideale oplossing aan te wijzen. Het eindresultaat kan ook met andere alternatieven bereikt worden.
De huidige versie van het prototype vormt een sterke basis, maar biedt nog ruimte voor verbetering en uitbreiding. Om diepgaandere conclusies te kunnen trekken en om de relevante eigenschappen van de gebruikersinterface grondig te analyseren, moet het prototype van de laptop-demo gedurende een langere periode worden ingezet. Deze applicatie biedt een moderne, creatieve en intuïtieve weergave die interessant kan zijn voor bedrijven die op zoek zijn naar ondersteunende interfaces.
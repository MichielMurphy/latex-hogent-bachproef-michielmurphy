\chapter{\IfLanguageName{dutch}{Stand van zaken}{State of the art}}%
\label{ch:stand-van-zaken}
\section [Esthetiek, hiërarchie en voorkeur]{Esthetiek, visuele hiërarchie en\\gebruikersvoorkeur}
Esthetiek is door de jaren heen op verschillende manieren gedefinieerd. Volgens \textcite[12]{Noponen2017} kan esthetiek omschreven worden als “de wetenschap van hoe dingen via de zintuigen worden herkend”. Website-esthetiek speelt een cruciale rol bij het ondersteunen van de inhoud en functionaliteit van een website. Elk visueel element communiceert iets naar de gebruiker. Deze effecten mogen door webdesigners niet worden genegeerd; het is juist door inzicht te hebben in deze effecten dat communicatie kan worden beïnvloed.\smallskip\par
De functionaliteit van een website verwijst naar de gebruiksvriendelijke aspecten van de interface. Het belangrijkste doel van functionaliteit is het creëren van websites waar gebruikers snel en efficiënt de gewenste informatie kunnen terugvinden~\autocite{Thorlacius2007}. Het verkrijgen van de gewenste inhoud is op veel websites moeilijk. Dit probleem ontstaat wanneer een website niet aansluit bij de behoeften van de gebruikers, maar zich voornamelijk richt op de interne prioriteiten van de organisatie.\par Een andere oorzaak is dat de informatie op de website niet logisch gestructureerd is voor de gebruiker~\autocite{Bevan1997}. De manier waarop een gebruiker met een webpagina interacteert, wordt beïnvloed door de visuele uitstraling van de pagina~\autocite{Michailidou2008}. Een aantrekkelijke compositie trekt de aandacht van de gebruiker en brengt de boodschap snel en duidelijk over. Een betekenisvol contrast tussen schermelementen, ruimtelijke groeperingen en het uitlijnen van de elementen draagt bij aan deze aantrekkingskracht en zijn aspecten van visuele hiërarchie~\autocite{Bhaskar2011}.\smallskip\par
Websitehiërarchie speelt een centrale rol in het structureren van informatie op websites en beïnvloedt hoe gebruikers zich op een website navigeren~\autocite{Djonov2007}. Uit het onderzoek van \textcite{Urano2021} blijkt dat hiërarchie in de grafische indeling gebruikers kan sturen in een voorgeschreven volgorde. Webdesigners gebruiken deze techniek om de gebruikerservaring te sturen. Dit principe wordt bijvoorbeeld gebruikt in kranten, waarbij een kop met groot, vetgedrukt lettertype de aandacht van de lezer trekt. In dergelijke gevallen bepaalt de redacteur welke informatie het belangrijkst is. Door het gebruik van grootte, kleur, contrast, positionering en typografie kunnen websites op een soortgelijke manier hun gebruikersinterface optimaliseren en de bruikbaarheid verbeteren~\autocite{Raghavendra2024}. Een cruciaal element van visuele hiërarchie is de juiste prioritering van informatie. Gebruikers kunnen op die manier snel de gewenste informatie terugvinden~\autocite{Raghavendra2024}.\smallskip\par In dit onderzoek ligt de nadruk echter op het visualiseren en begrijpen van de prioriteiten van de gebruiker. Volgens~\textcite{Lee2010} is het essentieel om inzicht te verkrijgen in hoe gebruikers hun voorkeuren vormen. Gebruikersvoorkeur verwijst naar de keuze tussen alternatieven op basis van persoonlijke mening. Deze voorkeur is een weerspiegeling van de gevoelens en houding een opzichte van de website en beïnvloedt ook het gedrag van de gebruiker op die website~\autocite{Lee2010}.\par
\bigskip
De bruikbaarheid en navigeerbaarheid van de IC moeten worden verbeterd. Hieronder wordt ingegaan op beide begrippen en hun belang.
\section{Bruikbaarheid}
Bruikbaarheid, ook wel gebruiksvriendelijkhe\-id genoemd, kan omschreven worden als de mate waarin gebruikers een website als gemakkelijk ervaren om te gebruiken~\autocite{Dianat2019}. Volgens~\textcite{Dingli2014} kan bruikbaarheid beschouwd worden als een kwaliteitsattribuut.Het doel van bruikbaarheid is om gebruikers te helpen hun taken efficiënt uit te voeren. Voor gebruikers die weinig tijd hebben om een systeem te leren kennen en minder computerervaring hebben, is bruikbaarheid van groot belang~\autocite{Mazumder2014}.\smallskip\par Een eenvoudig ontwerp dat voldoet aan de behoeften van de gebruiker zorgt ervoor een doelgerichte gebruiker hun taken snel en moeiteloos kunnen voltooien en draagt bij aan de gebruiksvriendelijkheid van de website~\autocite{Pearson2007}. Dit is exact waar de focus op ligt in dit onderzoek: het ontdekken van de noden van de gebruiker en de interface hierop afstemmen. Volgens~\textcite{Zachrison2022} wordt de eerste indruk van de bruikbaarheid vastgesteld door het kijken naar de grafische interface zonder er daadwerkelijk gebruik van te maken. Dit wordt de pre-use usability of de bruikbaarheid voor gebruik genoemd. In deze studie zal een nieuwe gebruikersinterface ontworpen worden met een positieve pre-use ability.

\section{Navigeerbaarheid}
Navigeerbaarheid, of navigatiegemak, verwijst naar de mate van moeite en tijd die een gebruiker nodig heeft om de gewenste informatie op een interface te vinden. Het is een cruciaal element voor de kwaliteit van een website en essentieel om ervoor te zorgen dat de gebruiker zich in controle voelt~\autocite{Zachrison2022}. Eenvoudige, efficiënte, gebruiksgerichte en flexibele navigatie helpt de gebruiker bij het bereiken van zijn doelen~\autocite{Pearson2007}. Door de filters in de nieuwe gebruikersinterface te prioriteren en visueel aan te passen, kan de gebruiker snel vinden wat hij nodig heeft.\par
\bigskip
Doorheen de studie worden verschillende systemen besproken, waaronder een expertsysteem en een recommendersysteem. Beide hebben als doel gebruikers te ondersteunen bij het vinden van wat ze zoeken. Omdat dit nauw aansluit bij een deel van het doel van de studie, worden deze systemen hieronder verduidelijkt.

\section{Wat is een expertsysteem en hoe werkt het?}
In de literatuur zijn meerdere definities van een expertsysteem (ES) te vinden. In het onderzoek van~\textcite[2]{Castillo2012} wordt het als volgt gedefinieerd: “Een expertsysteem kan in brede zin worden gedefinieerd als een computersysteem (hardware en software) dat menselijke experts in een specifiek vakgebied simuleert.” Een ES kan fungeren als consultant en is in staat onderbouwde beslissingen te nemen in zowel voorspelbare als onzekere omstandigheden. Dergelijke systemen waren dominant in industriële en medische sectoren.

Volgens~\textcite{Erdani2012} behoort het productieregel-gebaseerde systeem tot een van de meest populaire typen expertsystemen. Productieregels bieden een eenvoudige maar krachtige manier om kennis te representeren. Elke regel bestaat uit twee delen: een conditie of voorwaarde en een actie of conclusie. Wanneer de voorwaarde waar is, wordt de actie uitgevoerd.\par
\medskip
\textbf{Voorbeeld van een productieregel:}
\begin{itemize}
    \item \textbf{Regel:} ALS er een vlam is, DAN is er vuur.
    \item \textbf{Betekenis:} Vuur ontstaat als gevolg van een vlam.
\end{itemize}

De werking van een ES wordt uitgelegd in het rapport van~\textcite{Castillo2012} aan de hand van het volgende scenario:
\begin{itemize}
    \item \textbf{Type ES:} Een medisch expertsysteem
    \item \textbf{Invoer:} Symptomen, testresultaten en andere relevante feiten van een patiënt
    \item \textbf{Besluitvormingsproces:}
    {Besluitvormingsproces:}
    \begin{itemize}
        \item Het expertsysteem gebruikt de invoer als aanwijzingen en doorzoekt de kennisbank naar informatie die de juiste ziekte kan identificeren.
        \item Het trekt een conclusie op basis van deterministische kennis (zekere informatie) of probabilistische kennis (onzekere informatie).
        \item Bij onduidelijke of vage invoer past het systeem onzekerheidspropagatie toe om tot een waarschijnlijke diagnose te komen.
        \end{itemize}
    \item \textbf{Uitvoer:} Een al dan niet correcte diagnose.
\end{itemize}

\subsection{Voor- en nadelen van een expertsysteem}
Uit de studie van~\textcite{Castillo2012} zijn de volgende drie voordelen aangehaald:
\begin{itemize}
    \item {Combineren van kennis van meerdere experts wat leidt tot een betrouwbaarder systeem}
    \item {Sneller en efficiënter antwoorden dan een menselijke expert}
    \item {Oplossen van problemen te complex voor een menselijke expert}
\end{itemize}

Expertsystemen kennen ook verschillende nadelen. In deze studie wordt er slechts stilgestaan bij het feit dat ze niet flexibel zijn en zich niet kunnen aanpassen aan onvoorziene omstandigheden~\autocite{Turban1988}.

\section{Wat is een inferentie-engine en hoe werkt het?}
Een inferentie-engine (IE) is een essentieel onderdeel van een expertsysteem (ES) en wordt beschouwd als het hart van het systeem. Dit component trekt conclusies door abstracte kennis toe te passen op concrete kennis~\autocite{Castillo2012}. Voor het maken van deze afleidingen kunnen verschillende technieken worden gebruikt. In deze studie worden twee algoritmen besproken: forward chaining en backward chaining.

\subsection{Forward Chaining}
Bij forward chaining doorloopt de inferentie-engine herhaaldelijk de set van regels. Bij elke regel wordt gecontroleerd of de voorwaarden voldoen aan de reeds bekende feiten. Als alle voorwaarden van een regel waar zijn, wordt de conclusie uit de regel toegevoegd aan de kennisbank. Vervolgens herhaalt het proces zich aangezien de toegevoegde conclusie een nieuwe voorwaarde kan worden voor andere regels. Het systeem begint telkens opnieuw bij de eerste regel, waardoor mogelijke nieuwe verbanden worden ontdekt. Het proces eindigt wanneer er geen nieuwe conclusies meer gevonden worden~\autocite{Neapolitan1986}.\par
\medskip
\textbf{Concreet voorbeeld:}
Een medisch expertsysteem heeft de volgende regel: \textit{ALS X een bacterie is, DAN kan X worden behandeld met antibiotica.}
\begin{itemize}
    \item Het systeem controleert eerst alle bekende feiten om te bepalen of X een bacterie is.
    \item Als dat het geval is, wordt de conclusie \textit{X kan worden behandeld met antibiotica} toegevoegd aan de kennisbank.
\end{itemize}

\subsection{Backward Chaining}
Backward chaining is het tegenovergestelde van forward chaining. In plaats van te starten met bekende feiten, begint het systeem bij een doel en werkt het terug naar de feiten die dit doel kunnen ondersteunen. Daarom wordt het ook wel een doelgestuurde strategie genoemd~\autocite{Neapolitan1986}. Het proces start met een vraag waarvoor bewijs wordt gezocht om deze te beantwoorden. Het algoritme controleert of het doel al in de kennisbank aanwezig is. Indien het doel reeds een feit is, wordt het doel bevestigd. Anders zoekt en selecteert het systeem regels die overeenkomen met het doel~\autocite{Xie¸ski2018}. \par
\medskip
\textbf{Concreet voorbeeld:}
Een medisch expertsysteem ontvangt de volgende vraag: \textit{Kan patiënt X worden behandeld met antibiotica?}
\begin{enumerate}
    \item Het systeem zoekt naar een regel die deze vraag kan beantwoorden.
    \item Het systeem doorzoekt de kennisbank en vindt de volgende regel: \textit{ALS X een bacterie is, DAN kan X worden behandeld met antibiotica.}
    \item Om deze regel toe te passen, moet het systeem bevestigen dat X een bacterie is. Er wordt dus opnieuw gezocht naar een relevante regel en de volgende regel wordt gevonden: \textit{ALS X positief test op een bacteriële infectie, DAN is X een bacterie.}
    \item Het systeem heeft geen testresultaten en vraagt aan de patiënt of hij/zij positief getest heeft op een dergelijke test.
    \item De patiënt gaf aan positief getest te zijn waardoor de voorwaarde \textit{X is een bacterie} bevestigd is.
    \item Het systeem voegt tenslotte de conclusie \textit{X kan worden behandeld met antibiotica} toe aan de feitenbasis.
\end{enumerate}


\section{Wat is een recommender systeem?}
Kort na de uitvinding van het World Wide Web onstond het recommender-systeem \autocite{Dong2022}. Zoals vermeld in de inleiding, behoren deze aanbevelingssystemen tot de informatiefilteringsystemen en worden ze gebruikt om te voorspellen welke voorkeur een gebruiker aan een item zou geven. Over het algemeen raden dergelijke systemen gebruikers items aan die mogelijk binnen hun interesses vallen~\autocite{Thorat2015}. Volgens de studie van~\textcite{Dong2022} is het recommender-systeem tegenwoordig een van de meest succesvolle webapplicaties voor het aanbevelen van inhoud, zoals nieuwsfeeds, muziek en e-commerceproducten. Eerdere studies en succesverhalen bewijzen dat deze systemen erin slagen grote hoeveelheden data om te zetten in waardevolle inzichten. Volgens~\textcite{Thorat2015} zijn er vier hoofdtypen aanbevelingstechnieken:
\begin{enumerate}
    \item Collaborative Filtering (CF)
    \item Content-Based Filtering (CBF)
    \item Demografische Filtering
    \item Hybride Filtering
\end{enumerate}
In deze studie worden enkel Collaborative Filtering en Content-Based Filtering besproken.

\subsection{Collaborative Filtering (CF)}
Volgens~\textcite{Thorat2015} is Collaborative Filtering de meeste gebruikte methode voor het ontwerpen van aanbevelingssystemen. De suggesties worden gegenereerd door de voorkeuren van een actieve gebruiker te vergelijken met die van andere gebruikers die vergelijkbare beoordelingen aan producten hebben gegeven. Deze methode richt zich uitsluitend op het gedrag van de gebruiker, zonder dat het systeem hoeft te weten wie de gebruiker is~\autocite{Koren2021}.
Er zijn twee soorten CF-technieken: memory-based en model-based. Enkel het memory-based CF is relevant voor dit onderzoek.

\subsubsection{Memory-based CF}
Om voorspellingen te genereren, gebruiken memory-based CF-algoritmen de volledige of een deelverzameling van de gebruikers-itemdatabase. Elke gebruiker met vergelijkbare interesses behoort tot een groep vergelijkbare mensen. Door de buren van een nieuwe of momenteel actieve gebruiker te identificeren, kan het algoritme een voorspelling doen over de voorkeuren voor nieuwe items die gebruiker. Het meeste gebruikte algoritme binnen deze techniek is k Nearest Neighbors (kNN). Ook binnen deze techniek zijn er twee soorten algoritmen: user-based en item-based~\autocite{Thorat2015}.\par
\medskip
\textbf{User-Based Algoritme (UBA)}\par
User-based modellen voorspellen de voorkeuren van een gebruiker \textit{u} door eerst gelijkaardige gebruikers te vinden op basis van eerdere voorkeuren. Vervolgens worden de voorkeuren van deze gelijkaardige gebruikers gebruikt om een voorspelling te maken voor een doelitem \textit{i} voor \textit{u}~\autocite{Dong2022}. Een nadeel van het UBA is het schaalbaarheidsprobleem bij een toenemend aantal gebruikers~\autocite{Thorat2015}.\par
\medskip
\textbf{Item-Based Algoritme (IBA)}\par
Item-based modellen voorspellen de voorkeur van een gebruiker voor een item i op basis van zijn voorkeuren voor soortgelijke items die hij positief heeft beoordeeld of gewaardeerd~\autocite{Thorat2015,Dong2022}. In tegenstelling tot het UBA schaalt het IBA beter~\autocite{Thorat2015}.

\subsection{Content-Based CF}
Content-based filtering-algoritmes proberen items aan te bevelen op basis van de gelijkenis met items die de gebruiker eerder heeft beoordeeld. Hierbij spelen de itembeschrijving en het profiel van de voorkeuren van de gebruiker een belangrijke rol. Bij deze methode is de tf-idf-representatie het meest gebruikte algoritme, maar hier wordt niet verder op ingegaan~\autocite{Thorat2015}.

% Tip: Begin elk hoofdstuk met een paragraaf inleiding die beschrijft hoe
% dit hoofdstuk past binnen het geheel van de bachelorproef. Geef in het
% bijzonder aan wat de link is met het vorige en volgende hoofdstuk.

% Pas na deze inleidende paragraaf komt de eerste sectiehoofding.

%Dit hoofdstuk bevat je literatuurstudie. De inhoud gaat verder op de inleiding, maar zal het onderwerp van de bachelorproef *diepgaand* uitspitten. De bedoeling is dat de lezer na lezing van dit hoofdstuk helemaal op de hoogte is van de huidige stand van zaken (state-of-the-art) in het onderzoeksdomein. Iemand die niet vertrouwd is met het onderwerp, weet nu voldoende om de rest van het verhaal te kunnen volgen, zonder dat die er nog andere informatie moet over opzoeken \autocite{Pollefliet2011}.
%
%Je verwijst bij elke bewering die je doet, vakterm die je introduceert, enz.\ naar je bronnen. In \LaTeX{} kan dat met het commando \texttt{$\backslash${textcite\{\}}} of \texttt{$\backslash${autocite\{\}}}. Als argument van het commando geef je de ``sleutel'' van een ``record'' in een bibliografische databank in het Bib\LaTeX{}-formaat (een tekstbestand). Als je expliciet naar de auteur verwijst in de zin (narratieve referentie), gebruik je \texttt{$\backslash${}textcite\{\}}. Soms is de auteursnaam niet expliciet een onderdeel van de zin, dan gebruik je \texttt{$\backslash${}autocite\{\}} (referentie tussen haakjes). Dit gebruik je bv.~bij een citaat, of om in het bijschrift van een overgenomen afbeelding, broncode, tabel, enz. te verwijzen naar de bron. In de volgende paragraaf een voorbeeld van elk.
%
%\textcite{Knuth1998} schreef een van de standaardwerken over sorteer- en zoekalgoritmen. Experten zijn het erover eens dat cloud computing een interessante opportuniteit vormen, zowel voor gebruikers als voor dienstverleners op vlak van informatietechnologie~\autocite{Creeger2009}.
%
%Let er ook op: het \texttt{cite}-commando voor de punt, dus binnen de zin. Je verwijst meteen naar een bron in de eerste zin die erop gebaseerd is, dus niet pas op het einde van een paragraaf.

%\begin{figure}
%  \centering
%  \includegraphics[width=0.8\textwidth]{grail.jpg}
%  \caption[Voorbeeld figuur.]{\label{fig:grail}Voorbeeld van invoegen van een figuur. Zorg altijd voor een uitgebreid bijschrift dat de figuur volledig beschrijft zonder in de tekst te moeten gaan zoeken. Vergeet ook je bronvermelding niet!}
%\end{figure}

%\begin{listing}
%  \begin{minted}{python}
%    import pandas as pd
%    import seaborn as sns
%
%    penguins = sns.load_dataset('penguins')
%    sns.relplot(data=penguins, x="flipper_length_mm", y="bill_length_mm", hue="species")
%  \end{minted}
%  \caption[Voorbeeld codefragment]{Voorbeeld van het invoegen van een codefragment.}
%\end{listing}

%\lipsum[7-20]

%\begin{table}
%  \centering
%  \begin{tabular}{lcr}
%    \toprule
%    \textbf{Kolom 1} & \textbf{Kolom 2} & \textbf{Kolom 3} \\
%    $\alpha$         & $\beta$          & $\gamma$         \\
%    \midrule
%    A                & 10.230           & a                \\
%    B                & 45.678           & b                \\
%    C                & 99.987           & c                \\
%    \bottomrule
%  \end{tabular}
%  \caption[Voorbeeld tabel]{\label{tab:example}Voorbeeld van een tabel.}
%\end{table}
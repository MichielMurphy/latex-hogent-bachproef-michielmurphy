\chapter{\IfLanguageName{dutch}{Stand van zaken}{State of the art}}%
\label{ch:stand-van-zaken}

De huidige gebruikersinterface is onaantrekkelijk en overrompelt de gebruikers door de ondoordachte positionering en visualisatie van componenten. Het faalt iets te communiceren naar de gebruiker. Hieronder wordt het belang van esthetiek en de visuele hiërarchie van een website aangehaald en welke invloed dit heeft op de gebruikersvoorkeur.
\subsection [Esthetiek, hiërarchie en voorkeur]{Esthetiek, visuele hiërarchie en\\gebruikersvoorkeur}
Esthetiek is door de jaren heen op verschillende manieren gedefinieerd. Esthetiek kan worden omschreven als “de wetenschap van hoe dingen via de zintuigen worden herkend”~\autocite[p.~12]{Noponen2017}. Website-esthetiek speelt een cruciale rol bij het ondersteunen van de inhoud en functionaliteit van een website. Elk visueel element communiceert iets naar de gebruiker. Deze effecten mogen door webdesigners niet worden genegeerd; het is juist door inzicht te hebben in deze effecten dat communicatie kan worden beïnvloed. 
De functionaliteit van een website verwijst naar de gebruiksvriendelijke aspecten van de interface. Het belangrijkste doel van functionaliteit is het creëren van websites waar gebruikers snel en efficiënt de gewenste informatie kunnen terugvinden~\autocite{Thorlacius2007}. Het verkrijgen van de gewenste inhoud is op veel websites moeilijk. Dit probleem ontstaat wanneer een website niet aansluit bij de behoeften van de gebruikers, maar zich voornamelijk richt op de interne prioriteiten van de organisatie. Een andere oorzaak is dat de informatie op de website niet logisch gestructureerd is voor de gebruiker~\autocite{Bevan1997}.
De manier waarop een gebruiker met een webpagina interacteert, wordt beïnvloed door de visuele uitstraling van de pagina ~\autocite{Michailidou2008}. Een aantrekkelijke compositie trekt de aandacht van de gebruiker en brengt de boodschap snel en duidelijk over. Een betekenisvol contrast tussen schermelementen, ruimtelijke groeperingen en het uitlijnen van de elementen draagt bij aan deze aantrekkingskracht en zijn aspecten van visuele hiërarchie\\ \autocite{Bhaskar2011}.

Websitehiërarchie speelt een centrale rol in het structureren van informatie op websites en beïnvloedt hoe gebruikers zich op een website navigeren~\autocite{Djonov2007}. Uit het onderzoek van \textcite{Urano2021} blijkt dat hiërarchie in de grafische indeling gebruikers kan sturen in een voorgeschreven volgorde. Webdesigners gebruiken deze techniek om de gebruikerservaring te sturen. Dit principe wordt bijvoorbeeld gebruikt in kranten, waarbij een kop met groot, vetgedrukt lettertype de aandacht van de lezer trekt. In dergelijke gevallen bepaalt de redacteur welke informatie het belangrijkst is. Door het gebruik van grootte, kleur, contrast, positionering en typografie kunnen websites op een soortgelijke manier hun gebruikersinterface optimaliseren en de bruikbaarheid verbeteren\\\autocite{Raghavendra2024}.

Een cruciaal element van visuele hiërarchie is de juiste prioritering van informatie. Gebruikers kunnen op die manier snel de gewenste informatie terugvinden~\autocite{Raghavendra2024}. In dit onderzoek ligt de nadruk echter op het visualiseren en begrijpen van de prioriteiten van de gebruiker. Volgens~\textcite{Lee2010} is het essentieel om inzicht te verkrijgen in hoe gebruikers hun voorkeuren vormen. Gebruikersvoorkeur verwijst naar de keuze tussen alternatieven op basis van persoonlijke mening. Deze voorkeur is een weerspiegeling van de gevoelens en houding een opzichte van de website en beïnvloedt ook het gedrag van de gebruiker op die website~\autocite{Lee2010}.

De bruikbaarheid en navigeerbaarheid van de IC moeten verbeterd worden. Hieronder worden beide begrippen toegelicht en het belang ervan. (anders verwoorden)
\subsection{Bruikbaarheid}
Bruikbaarheid, ook wel gebruiksvriendelijkhe\-id genoemd, kan omschreven worden als de mate waarin gebruikers een website als gemakkelijk ervaren om te gebruiken~\autocite{Dianat2019}. Volgens~\textcite{Dingli2014} kan bruikbaarheid beschouwd worden als een kwaliteitsattribuut.\\Het doel van bruikbaarheid is om gebruikers te helpen hun taken efficiënt uit te voeren. Voor gebruikers die weinig tijd hebben om een systeem te leren kennen en minder computerervaring hebben, is bruikbaarheid van groot belang~\autocite{Mazumder2014}. Een eenvoudig ontwerp dat voldoet aan de behoeften van de gebruiker zorgt ervoor een doelgerichte gebruiker hun taken snel en moeiteloos kunnen voltooien en draagt bij aan de gebruiksvriendelijkheid van de website~\autocite{Pearson2007}. Dit is exact waar de focus op ligt in dit onderzoek: het ontdekken van de noden van de gebruiker en de interface hierop afstemmen. Volgens~\textcite{Zachrison2022} wordt de eerste indruk van de bruikbaarheid vastgesteld door het kijken naar de grafische interface zonder er daadwerkelijk gebruik van te maken. Dit wordt de pre-use usability of de bruikbaarheid voor gebruik genoemd. In deze studie zal een nieuwe gebruikersinterface ontworpen worden met een positieve pre-use ability.

\subsection{Navigeerbaarheid}
Navigeerbaarheid, of navigatiegemak, verwijst naar de mate van moeite en tijd die een gebruiker nodig heeft om de gewenste informatie op een interface te vinden. Het is een cruciaal element voor de kwaliteit van een website en essentieel om ervoor te zorgen dat de gebruiker zich in controle voelt~\autocite{Zachrison2022}. Eenvoudige, efficiënte, gebruiksgerichte en flexibele navigatie helpt de gebruiker bij het bereiken van zijn doelen~\autocite{Pearson2007}. Door de filters in de nieuwe gebruikersinterface te prioriteren en visueel aan te passen, kan de gebruiker snel vinden wat hij nodig heeft.

% Tip: Begin elk hoofdstuk met een paragraaf inleiding die beschrijft hoe
% dit hoofdstuk past binnen het geheel van de bachelorproef. Geef in het
% bijzonder aan wat de link is met het vorige en volgende hoofdstuk.

% Pas na deze inleidende paragraaf komt de eerste sectiehoofding.

%Dit hoofdstuk bevat je literatuurstudie. De inhoud gaat verder op de inleiding, maar zal het onderwerp van de bachelorproef *diepgaand* uitspitten. De bedoeling is dat de lezer na lezing van dit hoofdstuk helemaal op de hoogte is van de huidige stand van zaken (state-of-the-art) in het onderzoeksdomein. Iemand die niet vertrouwd is met het onderwerp, weet nu voldoende om de rest van het verhaal te kunnen volgen, zonder dat die er nog andere informatie moet over opzoeken \autocite{Pollefliet2011}.
%
%Je verwijst bij elke bewering die je doet, vakterm die je introduceert, enz.\ naar je bronnen. In \LaTeX{} kan dat met het commando \texttt{$\backslash${textcite\{\}}} of \texttt{$\backslash${autocite\{\}}}. Als argument van het commando geef je de ``sleutel'' van een ``record'' in een bibliografische databank in het Bib\LaTeX{}-formaat (een tekstbestand). Als je expliciet naar de auteur verwijst in de zin (narratieve referentie), gebruik je \texttt{$\backslash${}textcite\{\}}. Soms is de auteursnaam niet expliciet een onderdeel van de zin, dan gebruik je \texttt{$\backslash${}autocite\{\}} (referentie tussen haakjes). Dit gebruik je bv.~bij een citaat, of om in het bijschrift van een overgenomen afbeelding, broncode, tabel, enz. te verwijzen naar de bron. In de volgende paragraaf een voorbeeld van elk.
%
%\textcite{Knuth1998} schreef een van de standaardwerken over sorteer- en zoekalgoritmen. Experten zijn het erover eens dat cloud computing een interessante opportuniteit vormen, zowel voor gebruikers als voor dienstverleners op vlak van informatietechnologie~\autocite{Creeger2009}.
%
%Let er ook op: het \texttt{cite}-commando voor de punt, dus binnen de zin. Je verwijst meteen naar een bron in de eerste zin die erop gebaseerd is, dus niet pas op het einde van een paragraaf.

\begin{figure}
  \centering
  \includegraphics[width=0.8\textwidth]{grail.jpg}
  \caption[Voorbeeld figuur.]{\label{fig:grail}Voorbeeld van invoegen van een figuur. Zorg altijd voor een uitgebreid bijschrift dat de figuur volledig beschrijft zonder in de tekst te moeten gaan zoeken. Vergeet ook je bronvermelding niet!}
\end{figure}

%\begin{listing}
%  \begin{minted}{python}
%    import pandas as pd
%    import seaborn as sns
%
%    penguins = sns.load_dataset('penguins')
%    sns.relplot(data=penguins, x="flipper_length_mm", y="bill_length_mm", hue="species")
%  \end{minted}
%  \caption[Voorbeeld codefragment]{Voorbeeld van het invoegen van een codefragment.}
%\end{listing}

\lipsum[7-20]

\begin{table}
  \centering
  \begin{tabular}{lcr}
    \toprule
    \textbf{Kolom 1} & \textbf{Kolom 2} & \textbf{Kolom 3} \\
    $\alpha$         & $\beta$          & $\gamma$         \\
    \midrule
    A                & 10.230           & a                \\
    B                & 45.678           & b                \\
    C                & 99.987           & c                \\
    \bottomrule
  \end{tabular}
  \caption[Voorbeeld tabel]{\label{tab:example}Voorbeeld van een tabel.}
\end{table}


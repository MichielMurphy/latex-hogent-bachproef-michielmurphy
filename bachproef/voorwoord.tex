%%=============================================================================
%% Voorwoord
%%=============================================================================

\chapter*{\IfLanguageName{dutch}{Woord vooraf}{Preface}}%
\label{ch:voorwoord}

In dit voorwoord wil ik graag stilstaan bij het traject dat ik heb afgelegd om een onderwerp te vinden, en mijn ervaringen delen met het schrijven van deze bachelorproef.
Ik heb eerder een graduaatsopleiding Programmeren afgerond en volg momenteel een verkort traject Toegepaste Informatica. Aangezien ik tijdens het graduaat al stage heb gelopen, was ik hiervoor vrijgesteld. Dat maakte de zoektocht naar een geschikt bachelorproefonderwerp echter niet eenvoudiger. Ik had namelijk geen stagebedrijf waarop ik kon terugvallen voor het kiezen van een onderwerp.
Eén docent uit het graduaat, de heer Luc Vervoort, is me bijzonder bijgebleven. Hij zet zich sterk in voor studenten en staat altijd klaar om te helpen waar hij kan. In het begin van het eerste semester nam ik contact met hem op, in de hoop dat hij mij via zijn uitgebreide netwerk uit de nood kon helpen. Enkele weken later stelde hij mij dit praktische, interessante en uitdagende onderwerp voor, in samenwerking met co-promotor de heer Wim Delvaux van de KU Leuven.
Allereerst wil ik mijn oprechte dank uitspreken aan de heer Luc Vervoort. Daarnaast ben ik ook mijn promotor, de heer Jan Willem, zeer dankbaar voor de constructieve en duidelijke feedback, het vertrouwen in mijn kunnen en de steun die hij bood op momenten van onzekerheid. Ook wil ik mijn co-promotor, de heer Wim Delvaux, bedanken voor zijn waardevolle suggesties en kritische blik, die de technische uitwerking van mijn project naar een hoger niveau hebben getild. Deze bachelorproef bouwt voort op een bestaand project van de KU Leuven. In dat kader wil ik mijn waardering uitspreken voor Dr. Simon Vandevelde, een van de grondleggers van dit project. Hij heeft me ondersteund bij het begrijpen van de bestaande code en bezorgde me minder complexe scenario’s, waardoor ik sneller inzicht kreeg in de werking van het systeem. Tot slot gaat mijn dank uit naar mijn vriendin Jolien, voor haar onvoorwaardelijke steun, geduld en aanmoediging tijdens deze periode.
Het schrijven en uitvoeren van deze bachelorproef was een waardevol leerproces dat me heeft uitgedaagd om mijn grenzen te verleggen en mijn praktische vaardigheden te versterken. Het heeft me ook geleerd om kritisch na te denken vóór ik een keuze maak.
Ik durf dan ook te zeggen dat ik trots ben op wat ik met deze scriptie heb bereikt.
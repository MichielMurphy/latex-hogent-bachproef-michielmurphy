%==============================================================================
% Sjabloon poster bachproef
%==============================================================================
% Gebaseerd op document class `a0poster' door Gerlinde Kettl en Matthias Weiser
% Aangepast voor gebruik aan HOGENT door Jens Buysse en Bert Van Vreckem

\documentclass[a0,portrait]{hogent-poster}
\graphicspath{{../poster/graphics/}}

% Info over de opleiding
\course{Bachelorproef}
\studyprogramme{toegepaste informatica}
\academicyear{2024-2025}
\institution{Hogeschool Gent, Valentin Vaerwyckweg 1, 9000 Gent}

% Info over de bachelorproef
\title{Het inzetten van een front-end softwarebibliotheek die een dataset op een interactieve dynamische en prioriteit gestuurde manier visualiseert om structuur en duidelijkheid te creëren voor de gebruikersinterface van de IDP-Z3 webIDE van de KU Leuven}
%\subtitle{Ondertitel (eventueel)}
\author{Michiel Murphy}
\email{michiel.murphy@student.hogent.be}
\supervisor{Jan Willem}
\cosupervisor{Wim Delvaux (KU Leuven)}

% Indien ingevuld, wordt deze informatie toegevoegd aan het einde van de
% abstract. Zet in commentaar als je dit niet wilt.
\specialisation{Full stack developer}
\keywords{Woordwolk visualisatie, bruikbaarheid, navigeerbaarhied, prioriteitsgestuurd, dynamische interface}
\projectrepo{https://github.com/MichielMurphy/bachproef-michielmurphy}

\begin{document}

\maketitle

\begin{abstract}

\end{abstract}

\begin{multicols}{2} % This is how many columns your poster will be broken into, a portrait poster is generally split into 2 columns

\section{Introductie}




\section{Experimenten}



\section{Sectie met figuur}

De {\LaTeX} figure-omgeving bepaalt zelf waar een afbeelding komt en dat is meestal niet op de plek in de tekst waar de figure-omgeving gedefinieerd wordt. Als je wilt forceren dat afbeeldingen toch in de flow van de tekst blijven, dan kan je dat zoals hieronder:


%\begin{center}
%    \captionsetup{type=figure}
%    \includegraphics[width=0.8\linewidth,height=0.3\textheight,keepaspectratio]{flowchart.png}
%    \captionof{figure}{De verschillende uitgevoerde fasen van de methodologie.}
%\end{center}

Let er wel op dat dit tot problemen met bladschikking kan leiden.

\section{Conclusies}

Don't underestimate the Force. Oh God, my uncle. How am I ever gonna explain this? I suggest you try it again, Luke. This time, let go your conscious self and act on instinct. Don't be too proud of this technological terror you've constructed. The ability to destroy a planet is insignificant next to the power of the Force.

\section{Toekomstig onderzoek}

I care. So, what do you think of her, Han? No! Alderaan is peaceful. We have no weapons. You can't possibly… I have traced the Rebel spies to her. Now she is my only link to finding their secret base.

Kid, I've flown from one side of this galaxy to the other. I've seen a lot of strange stuff, but I've never seen anything to make me believe there's one all-powerful Force controlling everything. There's no mystical energy field that controls my destiny. It's all a lot of simple tricks and nonsense. You are a part of the Rebel Alliance and a traitor! Take her away! 

\end{multicols}
\end{document}